%old stuff
%The effect of biased reproductive sex ratios on the $N_{e}$ for an autosomal gene can be estimated using the equation \citep{wright1931evolution}:

%\begin{equation}
%N_{e} = \frac{4N_{m}N_{f}}{N_{m}+N_{f}}\label{eq:Ne}
%\end{equation}

%where $N_{m}$ and $N_{f}$ are the numbers of reproducing males and females, respectively. Similarly, for an X-linked gene, where females contain two-thirds and males one-third of the alleles, the effective population size is given by:

%\begin{equation}
%N_{e{X}} = \frac{9N_{m}N_{f}}{4N_{m}+2N_{f}}\label{eq:NeX}
%\end{equation}

%And, for the male-limited Y chromosome:

%\begin{equation}
%N_{e{Y}} = \frac{N_{m}}{2}\label{eq:NeY}
%\end{equation}


%\begin{equation}
%\frac{N_{e_{X}}}{N_{e_{A}}} = \frac{9(N_{f}+N_{m})}{8(2N_{f}+N_{m})}\label{eq:X/A}
%\end{equation}

%and,

%\begin{equation}
%\frac{N_{e_{Y}}}{N_{e_{A}}} = \frac{9(N_{f}+N_{m})}{8(2N_{f}+N_{m})}\label{eq:X/A}
%\end{equation}


 %Given a female effective population size $N_{f}$, and a male effective population size $N_{m}$, the predicted effective population size for X-linked genes, relative to autosomes, is given by $N_{e_{X}}/N_{e_{A}} = 9(N_{f}+N_{m})/8(2N_{f}+N_{m})$ (Wright 1931). Similarly, for the Y chromosome, the effective population size is given by 1/4 the total $N_{e}$ or 1/2 the $N_{e}$ of males, $N_{m}$/2. Thus, with equal sex ratios, $N_{e_{X}}/N_{e_{A}} = 0.75$, and $N_{e_{Y}}/N_{e_{A}} = 0.25$. For populations with female-biased sex ratios, however, $N_{m}$ becomes small relative to $N_{f}$ and this ratio can become larger than 1, approaching 1.125 at the limit \citep{caballero1995}. With the level of nucleotide polymorphism maintained in a population is proportional to the product of the mutation rate and the effective population size: $\theta=4N_{e}\mu$ \citep{watterson1975,kimura1984}. Figure 1 shows the predicted effective population size ratios for autosomal, X-linked, and Y-linked genes (left) and the corresponding X/A and Y/A ratios of diversity (right) for a sex ratio bias ranging from $N_{f}/(N_{f}+N_{m})=0$ to $N_{f}/(N_{f}+N_{m})=1$.

%In particular, Wright's (1931) formula $N_{e}=4N_{f}N_{m}/(N_{f} + N_{m})$ gives the effective population size of a randomly mating population with separate sexes and a Poisson distribution of offspring number. For a female effective population size $N_{f}$, and a male effective population size $N_{m}$, the predicted effective population size for X-linked genes relative to autosomes is given by $N_{e_{X}}/N_{e_{A}} = 9(N_{f}+N_{m})/8(2N_{f}+N_{m})$ (Wright 1931). With equal sex ratios, $N_{e_{X}}/N_{e_{A}} = 0.75$. For populations with female-biased sex ratios, however, $N_{m}$ becomes small relative to $N_{f}$ and this ratio can become larger than 1, approaching 1.125 at the limit \citep{caballero1995}. For the Y chromosome, the effective population size is given by 1/4 the total $N_{e}$ or 1/2 the $N_{e}$ of males, $N_{m}$/2. Under neutrality, the level of nucleotide polymorphism maintained in a population is proportional to the product of the mutation rate and the effective population size: $\theta=4N_{e}\mu$ \citep{watterson1975,kimura1984}. Assuming equal neutral mutation rates for all genes and an equal number of reproducing males and females, Y-linked genes should therefore have 1/3 the polymorphism of X-linked genes, and 1/4 that of autosomal genes. Figure 1 shows the predicted effective population size ratios for autosomal, X-linked, and Y-linked genes (left) and the corresponding X/A and Y/A ratios of diversity (right) for a sex ratio bias ranging from $N_{f}/(N_{f}+N_{m})=0$ to $N_{f}/(N_{f}+N_{m})=1$.
