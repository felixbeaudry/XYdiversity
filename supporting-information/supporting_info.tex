\documentclass[9pt,twocolumn,twoside]{pnas-new}
% Remove the twocolumn option to create a single column SI file if required. 
% Use the lineno option to display guide line numbers if required.
% Note that the use of elements such as single-column equations
% may affect the guide line number alignment. 

\templatetype{pnassupportinginfo}

\title{Supporting Information}
\author{Hough et al.}

\doi{genetics.XXXXXXXXXX}

\begin{document}

\maketitle

\section*{Supporting Information (SI)}

This supporting information may also be accessed at gihub at https://github.com/houghjosh/XYdiversity

\subsection*{SI Figures}


\subsection*{SI Tables}
Supply Word, RTF, or LaTeX files (LaTeX files must be accompanied by a PDF with the same file name for visual reference); include only one table per file. Do not use tabs or spaces to separate columns in Word tables.

\begin{table}[tbhp!]
\centering
\caption{Comparison of the fitted potential energy surfaces and ab initio benchmark electronic energy calculations}
\begin{tabular}{lrrr}
Species & CBS & CV & G3 \\
\midrule
1. Acetaldehyde & 0.0 & 0.0 & 0.0 \\
2. Vinyl alcohol & 9.1 & 9.6 & 13.5 \\
3. Hydroxyethylidene & 50.8 & 51.2 & 54.0\\
\bottomrule
\end{tabular}

\addtabletext{nomenclature for the TSs refers to the numbered species in the table.}
\end{table}

\subsection*{SI Datasets}

Supply Excel (.xls), RTF, CSV, TXT, or PDF files. This file type will be published in raw format and will not be edited or composed. 

\subsection*{SI software}

\end{document}